\documentclass{beamer}
\usetheme{Warsaw}
\usetheme{CambridgeUS}
\usefonttheme{structuresmallcapsserif}
\usefonttheme{serif}
\useinnertheme{circles}

\setbeamertemplate{background canvas}[vertical shading][bottom=white,top=white]   
\setbeamercolor{math text}{fg=black!10!Lublue}
\setbeamercolor{block title}{bg=blue!40!white, fg=black}
%\setbeamertemplate{navigation symbols}{}
\setbeamerfont{frametitle}{size=\normalsize}

\definecolor{light-gray}{gray}{0.99}
\definecolor{light-blue}{rgb}{0.90,0.90,0.98}
\definecolor{light-yellow}{rgb}{0.95,0.95,0.10}
\definecolor{dark-green}{rgb}{0.10,0.50,0.10}
\definecolor{Lublue}{rgb}{.10,.10,.70}
\definecolor{links}{rgb}{0.05,0.05,0.95}
\hypersetup{colorlinks,linkcolor=,urlcolor=links}

\definecolor{UNAMoro}{rgb}{0.796, 0.671, 0.341} % (secondary)
\definecolor{UNAMblue}{rgb}{0.067, 0.129, 0.275} % (primary)

%\setbeamercolor{palette primary}{bg=UNAMblue,fg=white}
%\setbeamercolor{palette secondary}{bg=UNAMblue,fg=white}
\setbeamercolor{palette tertiary}{bg=UNAMblue,fg=white}
%\setbeamercolor{palette quaternary}{bg=UNAMblue,fg=white}
\setbeamercolor{structure}{fg=UNAMblue} % itemize, enumerate, etc
\setbeamercolor{section in toc}{fg=UNAMblue} % TOC sections
\setbeamercolor{subsection in toc}{fg=Lublue} % TOC sections
% Override palette coloring with secondary
\setbeamercolor{subsection in head/foot}{bg=UNAMoro,fg=UNAMblue}
\setbeamercolor{frametitle}{bg=UNAMoro,fg=UNAMblue}
\setbeamercolor{subtitle}{bg=UNAMoro,fg=UNAMblue}


\usepackage[utf8]{inputenc}
\usepackage[spanish]{babel}
\usepackage{amsmath}
\usepackage{amsfonts}
\usepackage{amssymb}
\usepackage{graphicx}
\usepackage{color}
\usepackage{listings}
\usepackage{hyperref}
\usepackage{wasysym}
\usepackage{alltt}
\usepackage{algorithmic}
\usepackage{cancel}
\usepackage[export]{adjustbox}
\usepackage{caption}
\usepackage{tikz}
\usepackage{booktabs}
\usetikzlibrary{arrows,backgrounds}
\tikzstyle{block}=[draw opacity=0.7,line width=1.4cm]

\graphicspath{{./Figuras/}{../Figuras/}}

%%%%%%%%%%%%%%%%%%%%%%%%%%%%%% User specified LaTeX commands.
\newcommand{\Vector}[1]{{\mathbf{#1}}}
\newcommand{\Tensor}[1]{\underline{\underline{\mathbf{#1}}}}
%%%%%%%%%%%%%%%%%%%%%%%%%%%%%%
% Definimos el punto decimal.
\spanishdecimal{.}
%%%%%%%%%%%%%%%%%%%%%%%%%%%%%%

% the beginning of each subsection:
\AtBeginSection[]
{
	\begin{frame}<beamer>{Contenido}
	\tableofcontents[currentsection]
\end{frame}
}

\title[PAPIME--PE101019]{Diferencias Finitas}   
\subtitle{Introducción}

\author[\copyright LMCS]{ \textcolor{UNAMblue}{Modelación computacional en las ciencias y las ingenierías como apoyo en el proceso enseñanza-aprendizaje\\(PAPIME-PE101019)}} 
\institute[IGEF--UNAM] 
{ 
{\small{Instituto de Geof\'isica}} \\ 
\vspace{0.15cm}
{\small{Universidad Nacional Aut\'onoma de M\'exico}} \\
\vspace{0.15cm}
\includegraphics[height=.85cm]{unamlogo.png} 
}

\date[2019--2021]{\\ {\tiny{Esta obra está bajo una 
\href{https://creativecommons.org/licenses/by-nc-sa/4.0/}{Licencia Creative Commons Atribución-NoComercial-CompartirIgual 4.0 Internacional.}}\\
\includegraphics[scale=0.20]{ccPyNoxtli.png}}}

% To uncover everything in a step-wise fashion:
%\beamerdefaultoverlayspecification{<+->}

\begin{document}

\begin{frame}
\titlepage
\end{frame}

\begin{frame}{Contenido}
\tableofcontents
\end{frame}

\section{Derivadas numéricas de primer orden}

\begin{frame}
	
{\small Diferencias Finitas (FD) es una t\'ecnica para aproximar derivadas que permite 
obtener soluciones numéricas a Ecuaciones Diferenciales Ordinarias (ODE) y Parciales (PDE).}

{\footnotesize \begin{itemize}
\item Desarrollada por Leonhard Euler en 1768. Una excelente referencia de este método es \cite{Leveque}. 
\item La idea es muy simple y se basa en la estimaci\'on de la derivada de una funci\'on mediante la raz\'on de dos diferencias, por ejemplo:
\end{itemize}}

{\scriptsize 
\begin{block}{Aproximación hacia adelante (\textit{Forward})}
\begin{columns}
\begin{column}[t]{0.37\textwidth}
Para una funci\'on $f(x)$, la derivada en el punto $x_o$ est\'a definida por:
\[
\frac{d f}{d x}\Big|_{x_o} =
\lim_{h \to 0} \frac{f(x_o+h) - f(x_o)}{h}
\]
En la gráfica observamos que la línea azul intenta aproximar a la
tangente de la curva $f(x)$ en $x=x_o$ (línea roja).
\end{column}
\begin{column}[t]{0.58\textwidth}
\includegraphics[scale=0.38,valign=t]{Forward.pdf}
\captionof{figure}{heading}
\end{column}
\end{columns}	
\end{block}
}
\end{frame}


\begin{frame}

La aproximaci\'on en FD se mejora reduciendo $h$. 
\begin{itemize}
	\item {\small Para una valor finito de $h$, se introduce un error, el cual tiende a cero cuando $h \to 0$.}
\end{itemize}

{\footnotesize 
\begin{block}{Reduciendo la $h$}
\begin{columns}[t]
	\begin{column}[t]{0.35\textwidth}

En la gráfica, la línea roja representa la derivada exacta (pendiente en el punto azul).

\strut

Las líneas de colores indican diferentes aproximaciones a la derivada para distintos valores de $h$.

\strut

Se observa que conforme $h$ se hace más pequeña, la aproximación a la derivada es cada vez mejor.
	\end{column}
	\begin{column}[t]{0.60\textwidth}
\includegraphics[scale=0.40, valign=t]{Forward_hs.pdf}

	\[
	\frac{d f}{d x}\Big|_{x_o} =
	\lim_{h \to 0} \frac{f(x_o+h) - f(x_o)}{h}
	\]
	\end{column}
\end{columns}
\end{block}
}

\end{frame}

\subsection{Forward, Backward, Centered}

\begin{frame}

\begin{itemize}
	\item Para analizar el error hagamos una expansión en Series de Taylor
	alrededor del punto $x_o$: 
%	
	\begin{eqnarray*}
 		f(x) & = & f(x_o) + f^\prime(x_o)(x - x_o) + f^{\prime\prime}(x_o)\frac{(x - x_o)^2}{2!} + \\
      		 & & f^{\prime\prime\prime}(x_o)\frac{(x - x_o)^3}{3!} + \dots + f^{(n)}(x_o)\frac{(x - x_o)^n}{n!} + \\
   			 & & R_n(x)
	\end{eqnarray*}
%	
    $R_n(x)$ representa el residuo, véase el Teorema de Taylor en \cite{Burden}.

\strut
	
	\item Ahora evalúamos la serie anterior en $x = x_o+h$:
	\begin{eqnarray*}
		f(x_o+h) & = & f(x_o) + h f^\prime(x_o) + \frac{h^2}{2!}f^{\prime\prime}(x_o)
		+ \frac{h^3}{3!}f^{\prime\prime\prime}(x_o) + \mathcal{O}(h^4).	
	\end{eqnarray*}
	donde $\mathcal{O}(h^4)$ representa términos de orden mayores o iguales a $h^4$.
\end{itemize}
\end{frame}

\begin{frame}
\begin{itemize}
	\item La expansión anterior se puede reescribir como sigue:
	\[
	\underbrace{\frac{f(x_o+h)-f(x_o)}{h}}_{D_+f(x_o)} = f^\prime(x_o) + \frac{h}{2!}f^{\prime\prime}(x_o)
	+ \frac{h^2}{3!}f^{\prime\prime\prime}(x_o) + \mathcal{O}(h^3). 
	\]
	y restando $f^\prime(x_o)$ de ambos lados obtenemos el error absoluto de la aproximación:
	\[ 
	D_+f(x_o) - f^\prime(x_o) = \underbrace{\frac{h}{2!}f^{\prime\prime}(x_o)
		+ \frac{h^2}{3!}f^{\prime\prime\prime}(x_o) + \mathcal{O}(h^3)}_{\mathcal{O}(h)}
	\]
Esta es una aproximación de primer orden $\mathcal{O}(h)$.
\end{itemize}
\end{frame}


\begin{frame}
\begin{itemize}
\item Similarmente, si evaluamos la expansión en Series de Taylor en $x = x_o-h$ obtenemos:
\begin{eqnarray*}
	f(x_o-h) & = & f(x_o) - h f^\prime(x_o) + \frac{h^2}{2!}f^{\prime\prime}(x_o)
	- \frac{h^3}{3!}f^{\prime\prime\prime}(x_o) + \mathcal{O}(h^4). 	
\end{eqnarray*}
 y por lo tanto:
	\[
	\underbrace{\frac{f(x_o)-f(x_o-h)}{h}}_{D_-f(x_o)} = f^\prime(x_o) - \frac{h}{2!}f^{\prime\prime}(x_o)
	+ \frac{h^2}{3!}f^{\prime\prime\prime}(x_o) + \mathcal{O}(h^3). 
	\]
	
 cuyo error absoluto es:
	\[ 
	D_-f(x_o) - f^\prime(x_o) = \underbrace{-\frac{h}{2!}f^{\prime\prime}(x_o)
		+ \frac{h^2}{3!}f^{\prime\prime\prime}(x_o) + \mathcal{O}(h^3)}_{\mathcal{O}(h)}
	\]
%
Esta aproximación también es de primer orden $\mathcal{O}(h)$.
\end{itemize}
\end{frame}

\begin{frame}
\begin{itemize}
\item Si restamos las expansiones evaluadas en $x_o+h$ y $x_o-h$ obtenemos:
{\footnotesize 
	\begin{eqnarray*}
	f(x_o+h) & = & f(x_o) + h f^\prime(x_o) + \frac{h^2}{2!}f^{\prime\prime}(x_o)
	+ \frac{h^3}{3!}f^{\prime\prime\prime}(x_o) + \mathcal{O}(h^4) \\ -\quad \qquad \qquad & &\\
	f(x_o-h) & = & f(x_o) - h f^\prime(x_o) + \frac{h^2}{2!}f^{\prime\prime}(x_o)
	- \frac{h^3}{3!}f^{\prime\prime\prime}(x_o) + \mathcal{O}(h^4) \\
	\hline
	f(x_o+h) - f(x_o-h) & = & 2 h f^\prime(x_o) + 2\frac{h^3}{3!}f^{\prime\prime\prime}(x_o) + \mathcal{O}(h^4)
\end{eqnarray*}
\[
\Longrightarrow \underbrace{\frac{f(x_o+h) - f(x_o-h)}{2h}}_{D_0 f(x_o)} = f^\prime(x_o) + \frac{h^2}{3!}f^{\prime\prime\prime}(x_o) + \mathcal{O}(h^3). 
\]
}
\item Observe que en este caso el error absoluto es de orden cuadrático:
{\small
\[
D_0 f(x_o) - f^\prime(x_o)  = \underbrace{\frac{h^2}{3!}f^{\prime\prime\prime}(x_o) + \mathcal{O}(h^3)}_{\mathcal{O}(h^2)}. 
\]
}
\end{itemize}
\end{frame}

\begin{frame}{Forward, Backward, Centered}

%Tenemos tres aproximaciones para la primera derivada en $x_o$:
\begin{footnotesize}
\begin{itemize}
	\item Hacia adelante (Forward): $\displaystyle D_+f(x_o) = \frac{f(x_o+h)-f(x_o)}{h}$
	\item Hacia atrás (Backward): $\displaystyle D_-f(x_o) = \frac{f(x_o)-f(x_o-h)}{h}$
	\item Centradas (Centered): $\displaystyle D_0 f(x_o) = \frac{f(x_o+h) - f(x_o-h)}{2h}$
\end{itemize}
\end{footnotesize}

\begin{footnotesize}
\begin{block}{Forward, Backward, Centered}
\begin{columns}[t]
\begin{column}{0.35\textwidth}
La línea azul y la línea naranja son aproximaciones de primer orden, Forward y Backward respectivamente.

\strut

La línea verde es una aproximación de segundo orden, Centrada, y su pendiente es muy similar a la
de la línea roja (derivada exacta).
\end{column}
\begin{column}[t]{0.60\textwidth}
\includegraphics[scale=0.40,valign=t]{AllDerivatives.pdf}
\end{column}
\end{columns}
\end{block}
\end{footnotesize}

\end{frame}

\subsection{Aproximaciones con más puntos}

\begin{frame}{Aproximaciones con más puntos}

\begin{itemize}
	\item Las aproximaciones $D_+$ y $D_-$ se hacen usando un punto a la derecha y un punto a la izquierda de $x_o$, respectivamente. En 
	ambos casos se obtiene una precisión de $\mathcal{O}(h)$.
	\item En el caso de $D_0$ se usan dos puntos, uno a la izquierda y otro a la derecha de $x_o$ y se obtiene una precisión de $\mathcal{O}(h^2)$.
	\item Es posible usar más puntos en la aproximación, pues entre más puntos se usen, la aproximación será mejor.
	\item Para ello usaremos la siguiente  la notación con subíndices:
{\small 
\[	
\begin{array}{cccccccccc}
x_o   \equiv  x_{i}, & f(x_{i}) \equiv f_{i}, & \frac{d f(x_{i})}{d x} \equiv f^\prime_{i}, & \dots \\
x_o+h \equiv  x_{i+1}, & f(x_{i+1}) \equiv f_{i+1}, & \frac{d f(x_{i+1})}{d x} \equiv f^\prime_{i+1}, & \dots \\
x_o-h \equiv  x_{i-1}, & f(x_{i-1}) \equiv f_{i-1}, & \frac{d f(x_{i-1})}{d x} \equiv f^\prime_{i-1}, & \dots \\
x_o+2h \equiv x_{i+2}, & f(x_{i+2}) \equiv f_{i+2}, & \frac{d f(x_{i+2})}{d x} \equiv f^\prime_{i+2}, & \dots \\
x_o-2h \equiv x_{i-2}, & f(x_{i-2}) \equiv f_{i-2}, & \frac{d f(x_{i-2})}{d x} \equiv f^\prime_{i-2}, & \dots \\
 & \vdots
\end{array}
\]
}
\end{itemize}
\end{frame}

\begin{frame}{Aproximaciones con más puntos}

Supongamos que deseamos hacer una aproximación como la que sigue:
% 
\[
f^\prime_i=A f_i + B f_{i-1}+C f_{i-2} + \mathcal{O}(h^2)
\]
%
En la fórmula anterior deseamos aproximar $f^\prime_i$ usando $f_i$, $f_{i-1}$ y $f_{i-2}$ y para ello debemos
encontrar los coeficientes $A$, $B$ y $C$.
{\footnotesize 
\begin{block}{Ejemplo 1: $D_{-2}$ (dos puntos a la izquierda de $x_o$)}
\begin{columns}[t]
\begin{column}{0.35\textwidth}
En la gráfica de la derecha se marca con azul el punto donde se desea realizar la aproximación
y con naranja los puntos auxiliares. 

\strut

Los dos puntos naranjas y el punto azul (tres puntos) serán usados para encontrar los coeficientes
$A$, $B$ y $C$.
\end{column}
\begin{column}[t]{0.6\textwidth}
\includegraphics[scale=0.40,valign=t]{FDPoints_L.pdf}
\end{column}
\end{columns}
\end{block}
}
\end{frame}

\begin{frame}{Aproximaciones con más puntos}
\begin{itemize}
\item Para encontrar los coeficientes $A$, $B$ y $C$, usamos la expansión en series de Taylor alrededor de $x_o$
y la evaluamos en $x_o - h = x_{i-1}$ y en $x_o - 2h = x_{i-2}$:
\begin{eqnarray*}
	f_{i-1} & = & f_i + (-h) f^\prime_i + \frac{(-h)^2}{2!} f^{\prime\prime}_i + 
	\frac{(-h)^3}{3!} f^{\prime\prime}_i + \mathcal{O}(h^3) \\
	f_{i-2} & = & f_i + (-2h) f^\prime_i + \frac{(-2 h)^2}{2!} f^{\prime\prime}_i + 
	\frac{(-2 h)^3}{3!} f^{\prime\prime}_i + \mathcal{O}(h^3) 
\end{eqnarray*}
\noindent Nótese que $(x_{i-1} - x_{i}) = -h$ y $(x_{i-2} - x_{i}) = -2h$

\item Sustituimos estas ecuaciones en la fórmula: 
\[
f^\prime_i= A f_i + B f_{i-1} + C f_{i-2}
\]
y resolvemos el sistema lineal resultante para obtener los coeficientes.
\end{itemize}
\end{frame}

\begin{frame}{Aproximaciones con más puntos}
\begin{itemize}
	\item Entonces:
\begin{eqnarray*}
f^\prime_i & = &  A f_i + 
 B \left(f_i -h f^\prime_i + \frac{h^2}{2} f^{\prime\prime}_i - 
 \frac{h^3}{6} f^{\prime\prime}_i + \dots \right) + \\
 & & C \left(f_i -2h f^\prime_i + \frac{4 h^2}{2} f^{\prime\prime}_i -
 \frac{8 h^3}{6} f^{\prime\prime}_i + \dots \right) \\
f^\prime_i & = &  (A + B + C) f_{i} - (B + 2C) h f^\prime_i + \left(\frac{B}{2} + 2C\right) h^2 f^{\prime\prime}_i + \mathcal{O}(h^3)
\end{eqnarray*}
Nótese que no necesitamos más términos para poder encontrar los tres coeficientes.
\item Para que ambos lados de la ecuación anterior sean iguales se debe cumplir lo siguiente:
\[
\begin{array}{ccc}
\begin{array}{rcc}
A + B + C & = & 0 \\
B + 2C & = & -\frac{1}{h} \\
B + 4C & = & 0 
\end{array} 
& \Longrightarrow &
\left(
\begin{array}{ccc}
1 & 1 & 1 \\
0 & 1 & 2 \\
0 & 1 & 4
\end{array}
\right)
\left(
\begin{array}{c}
A \\ B \\ C 
\end{array}
\right) = 
\left(
\begin{array}{c}
0 \\ -\frac{1}{h} \\ 0 
\end{array}
\right)

\end{array}
\]
\end{itemize}
\end{frame}

\begin{frame}{Aproximaciones con más puntos}

\begin{itemize}
	\item Resolviendo el sistema lineal anterior obtenemos:
\[
\begin{array}{ccc}
A = \frac{3}{2h} & B = -\frac{2}{h} & C = \frac{1}{2h}
\end{array}
\]
	\item Por lo tanto:
\[
\boxed{f^\prime(x_i) = \frac{3 f_i - 4 f_{i-1} + f_{i-2}}{2h} = D_{-2} f(x_i)}
\]

Observe que hemos llamado al resultado $D_{-2}f(x_i)$ que indica que se usan dos puntos a la izquierda de $x_i$.
El orden de esta aproximación es $\mathcal{O}(h^2)$. 
\end{itemize}

\end{frame}

\begin{frame}{Aproximaciones con más puntos}
%

{\footnotesize 
\begin{block}{Ejemplo 2: $D_{+2}$ (dos puntos a la derecha de $x_o$)}
\begin{columns}[t]
\begin{column}[t]{0.35\textwidth}
Otra aproximación:
% 
\[
f^\prime_i = A f_i + B f_{i+1} + C f_{i+2} + \mathcal{O}(h^2)
\]
Ahora se desean usar dos puntos a la derecha de $i$, observe los puntos color naranja de la figura.
\end{column}
\begin{column}[t]{0.6\textwidth}
\includegraphics[scale=0.40, valign=t]{FDPoints_R.pdf}
\end{column}
\end{columns}

Es posible encontrar los coeficientes de $A$, $B$ y $C$ de esta aproximación siguiendo la estrategia
explicada antes.
\end{block}
}
%
\end{frame}

\begin{frame}

\begin{block}{Ejemplo 3.}
{\scriptsize 
Sea $u(x) = \sin(x)$. (1) Aproximar $u^{\prime}(x) = \cos(x)$ en $x_o = 1$, es decir $\cos(1) \approx 0.5403$, 
usando $D_-$, $D_+$, $D_0$, $D_{-2}$, $D_{+2}$ y $D_3  = \frac{1}{6h}[2u_{i+1} + 3u_i - 6u_{i-1} + u_{i-2}]$ 
y calcular el error absoluto: $|\cos(1) - D_i|$ para $i= -, + , 0, -2, +2, 3$.
%

\textbf{Solución:}}
{\tiny 
\begin{tabular}{rrrrrrr}
	\toprule
	$h$ &     $D_+$ &     $D_-$ &    $D_0$ &      $D_{+2}$ &   $D_{-2}$ &         $D_3$ \\
	\midrule
	0.100 &  0.042939 &  0.041138 &  9.000537e-04 &  2.004728e-03 &  1.584693e-03 &  6.820693e-05 \\
	0.050 &  0.021257 &  0.020807 &  2.250978e-04 &  4.761431e-04 &  4.235730e-04 &  8.649142e-06 \\
	0.010 &  0.004216 &  0.004198 &  9.004993e-06 &  1.821981e-05 &  1.779908e-05 &  6.994130e-08 \\
	0.005 &  0.002106 &  0.002101 &  2.251257e-06 &  4.528776e-06 &  4.476184e-06 &  8.754000e-09 \\
	0.001 &  0.000421 &  0.000421 &  9.005045e-08 &  1.803108e-07 &  1.798903e-07 &  6.997947e-11 \\
	\bottomrule
\end{tabular}}
%
\begin{columns}
\begin{column}{0.50\textwidth}
$$\includegraphics[scale=0.25]{LogLogFDM}$$
\end{column}
\begin{column}{0.45\textwidth}
{\scriptsize Observamos, en la gráfica $log-log$ que el error se comporta como:
$$\log(E(h)) \approx p \log h + \log|C|$$
entonces: 
$$E(h) \approx C \, h^p$$
es decir, \textcolor{blue}{la pendiente de cada l\'inea recta es el orden de la aproximaci\'on}}.
\end{column}
\end{columns}

aadsdasd

\end{block}

\end{frame}

\section{Ejercicio 1.}

\begin{frame}
	
{\footnotesize 	
\begin{exampleblock}{Ejercicio 1.}
\begin{enumerate}
	\item Calcular los coeficientes $A$, $B$ y $C$ para $D_{+2}$.
	\item Demostrar que el error absoluto de las aproximaciones $D_{-2}$ y $D_{+2}$ es $\mathcal{O}(h^2)$.
	\item Calcular los coeficiente de la aproximación $D_{3} f(x) = A f_{i+1} + B f_{i} + C f_{i-1} + D f_{i-2}$
	y demostrar que el orden de esta aproximación es $\mathcal{O}(h^3)$.
	\item Reproducir la tabla y la gráfica del ejemplo 3, para ello realice los siguientes pasos:
	\begin{enumerate}[a]
		\item {\scriptsize Abra el notebook \texttt{E01\_DerivadasNum.ipynb} del repositorio \texttt{Mixbaal}
		(\url{https://github.com/luiggix/Mixbaal}).}
		\item {\scriptsize Observe que ya se encuentran implementados los casos para $D_{+}$, $D_{+}$ y $D_{0}$.
		Realice las implementaciones para las aproximaciones faltantes y obtenga la tabla y la gráfica final.}
	\end{enumerate}
\end{enumerate}
\end{exampleblock}
}
	
\end{frame}

\section{Derivadas numéricas de orden superior}

\begin{frame}{Derivadas de orden superior}
	
	\begin{itemize}
		\item {\small Es posible encontrar aproximaciones a derivadas de orden mayor a uno. 
		Por ejemplo, para orden $2$, se escribe la expansión en series de Taylor de $f(x)$,
		se evalúa en $x_o + h$ y en $x_o - h$ y luego se hace la suma:}
%	
{\footnotesize 
	\begin{eqnarray*}
		f(x_o+h) & = & f(x_o) + h f^\prime(x_o) + \frac{h^2}{2!}f^{\prime\prime}(x_o)
		+ \frac{h^3}{3!}f^{\prime\prime\prime}(x_o) + \mathcal{O}(h^4) \\ +\quad \qquad \qquad & &\\
		f(x_o-h) & = & f(x_o) - h f^\prime(x_o) + \frac{h^2}{2!}f^{\prime\prime}(x_o)
		- \frac{h^3}{3!}f^{\prime\prime\prime}(x_o) + \mathcal{O}(h^4) \\
		\hline
		f(x_o+h) + f(x_o-h) & = & 2 f(x_o) + h^2 f^{\prime\prime}(x_o) + \mathcal{O}(h^4)
	\end{eqnarray*}
	\[
	\Longrightarrow \underbrace{\frac{f(x_o+h) - 2 f(x_o) + f(x_o-h)}{h^2}}_{D^2 f(x_o)} = f^{\prime\prime}(x_o) + \frac{1}{12} h^2 f^{\prime\prime\prime\prime}(x) + \mathcal{O}(h^4). 
	\]
%		
	cuyo error absoluto es:
		\[ 
		D^2 f(x) - f^{\prime\prime}(x) = \frac{1}{12} h^2 f^{\prime\prime\prime\prime}(x) + \mathcal{O}(h^4)
		\]
}		
{\small Como se puede observar, esta aproximación es de orden $\mathcal{O}(h^2)$.}
		
	\end{itemize}
\end{frame}

\begin{frame}{Derivadas de orden superior}
	
	\begin{itemize}
		\item Otra manera de obtener las aproximaciones para derivadas de orden mayor que $1$ es aplicando 
		repetidamente las diferencias de primer orden.
		\item Por ejemplo, se puede ver que:
		%
		\[
		D^2 f(x) = D_+ D_- f(x)
		\]
		
		pues 
		
		\begin{eqnarray*}
			D_+(D_-f(x)) & = & \frac{1}{h} \left[D_+ f(x+h) - D_-f(x) \right] \\
			& = & \frac{1}{h} \left[\left(\frac{f(x+h) - f(x)}{h} \right) - \left(\frac{f(x) - f(x-h)}{h} \right)\right] \\
			& = &\frac{f(x+h) - 2f(x) + f(x-h)}{h^2} = D^2 f(x)
		\end{eqnarray*}
		
	\end{itemize}
\end{frame}

\begin{frame}{Derivadas de orden superior}
	
	\begin{itemize}
		\item Alternativamente también es posible hacer:
		
		
		\[
		D^2 f(x) = D_- D_+ f(x)
		\]
		
		o
		
		\[
		D^2 f(x) = \hat{D}_0 (\hat{D}_0 f(x))
		\]
		
		Donde $\hat{D}_0$ es una aproximación de primer orden en diferencias centradas usando $h/2$.
		Verifique que ambas aproximaciones son válidas.
		
		\item Usando la notación con subíndices se puede escribir:
		
		\begin{equation}\label{eq:segundaDer}
		\boxed{f^{\prime\prime}(x) = \frac{d^2 f(x)}{d x} \approx \frac{f_{i+1} - 2 f_{i} + f_{i-1}}{h^2} = D^2 f(x)}
		\end{equation}
	\end{itemize}	
\end{frame}

\section<presentation>{Referencias}

\begin{frame}[allowframebreaks]
	%\frametitle<presentation>{Bibliograf\'{\i}a}
	
\begin{thebibliography}{10}
		
{\footnotesize 

% BOOKS 
\beamertemplatebookbibitems


\bibitem{Leveque}
[1] R.J. Leveque,
\newblock {\em Finite Difference Method for Ordinary and Partial Differential Equations: Steady State and Time-Dependent Problems },
\newblock {Society for Industrial and Applied Mathematics (SIAM), Philadelphia}, \textbf{2007}.

\bibitem{Saad}
[2] Y. Saad
\newblock {\em Iterative Methods for Sparse Linear Systems}.
\newblock PWS/ITP 1996.
\newblock {Online: 
	\textsf{http://www-users.cs.umn.edu/\textasciitilde saad/books.html}, 
	\textbf{2000}}

\bibitem{Burden}
[3]  Richard Burden and J. Douglas Faires
\newblock{\em Numerical Analysis}
\newblock Cengage Learning; 9 edition (August 9, \textbf{2010})

% ARTICULOS    
\beamertemplatearticlebibitems

%\bibitem{Delacruz2011}
%[7]   L. M. de la Cruz,
%\newblock Flujo en una y dos fases en medios porosos: modelos matemáticos, numáricos y computacionales,
%\newblock {\em Reportes técnicos del Instituto de Geofísica}, 2012-4, Agosto \textbf{2012}.


}
		
\end{thebibliography}
\end{frame}

\section{Créditos}

\begin{frame}{Créditos}
	
	\begin{center}
		\textbf{Dr. Luis M. de la Cruz Salas} \\
		\vspace{0.5cm}
		{\small{Departamento de Recursos Naturales}} \\
		\vspace{0.15cm}
		{\small{Instituto de Geof\'isica}} \\ 
		\vspace{0.15cm}
		{\small{Universidad Nacional Aut\'onoma de M\'exico}} \\
		\vspace{0.15cm}
		\includegraphics[height=.85cm]{unamlogo.png} \\
		\vspace{1.15cm}
		{\scriptsize{Trabajo realizado con el apoyo del Programa UNAM-DGAPA-PAPIME PE101019 \\		
		\includegraphics[scale=0.20]{ccPyNoxtli.png} }}
	\end{center}
	
\end{frame}


\end{document}
	